\documentclass[10pt,aspectratio=54, handout]{beamer}
%\usefonttheme[onlymath]{serif}
\usepackage{graphicx}
\usetheme{Berlin}
\usepackage{polyglossia}
\setdefaultlanguage[variant=indonesian]{malay}


\newcommand{\floor}[1]{\lfloor #1 \rfloor}
\newcommand{\ceil}[1]{\lceil #1 \rceil}
\author{Haikal Isa Al Mahdi}
\title{Fungsi}

\begin{document}
  \begin{frame}
    \maketitle
  \end{frame}
  
  \frame{
    \frametitle{Daftar Isi}
    \tableofcontents
  }
  
  \section{Fungsi yang Kita Kenal}
  \label{sec:Fungsi yang Kita Kenal}
  
  \begin{frame}{Fungsi yang Kita Kenal}
    Kita pakai definisi berikut
    \begin{block}{Definisi IO}
      Fungsi $f\,(x)$ mengambil suatu $x$ sebagai input dan menghasilkan nilai keluaran.
    \end{block}
    
    \begin{exampleblock}{Contoh}
      Diketahui $f(x) = x \bmod 12$ \\~\\
      $f(9)$ akan menghasilkan keluaran $9$, namun $f(15)$ menghasilkan keluaran $3$.
    \end{exampleblock}
  \end{frame}
  
  \begin{frame}{Kenapa Fungsi}
    Ketahuilah bahwa fungsi termasuk hal yang mudah untuk dikerjakan dalam osk informatika. Ekspresi dalam bentuk fungsi lebih dapat dicerna ketimbang dalam bentuk kode sumber maupun pseudocode.
    \\
    Substitusi itu penting
  \begin{exampleblock}{Contoh Soal}
    Diketahui $f(x^2+2) = x$. Tentukan $f(x)$.
  \end{exampleblock}
  
  \begin{block}{Jawaban}
    substitusikan $\sqrt{x-2}$ ke dalam fungsi. Dengan ini, $f(x) = \sqrt{x-2}$.
  \end{block}
  \end{frame}
  
  
  \section{Rekursi}
  \label{sec:Rekursi}
  
  \subsection{Satu Variabel}
  \label{sub:Satu Variabel}
  
  \begin{frame}{Fungsi Rekursi}
  \begin{columns}[c]
    \column{0.4\textwidth}
  Materi tentang rekursi terkadang keluar di osk informatika. Fungsi rekursi adalah fungsi yang memanggil diri sendiri. Namun, agar tak terjadi pemanggilan yang tak terhingga, terdapat nilai dasar (base case).
    
    
    \column{0.6\textwidth}
    \begin{exampleblock}{Contoh}
      Fungsi faktorial secara rekursif didefinisikan sebagai berikut:
      \begin{itemize}
        \item $f(x) = xf(x-1)$
        \item $f(0) = f(1) = 1$
      \end{itemize}
      Cara kerjanya sebagai berikut: Misalkan kita ingin mencari nilai $f(4)$. Prosesnya sebagai berikut:
      \begin{itemize}
        \item $f(4) = 4\cdot f(3)$
        \item $f(4) = 4\cdot 3\cdot f(2)$
        \item $f(4) = 4\cdot 3\cdot 2\cdot f(1)$
        \item $f(4) = 4\cdot 3\cdot 2\cdot 1$
        \item $\boxed{f(4) = 24}$
      \end{itemize}
      
    \end{exampleblock}
  
  \end{columns}
  
  \end{frame}
  
  \subsection{Multivariabel}
  \label{sub:Multivariabel}
  
  \begin{frame}{Fungsi dengan Banyak Variabel}
  \begin{columns}[c]
    \column{0.5\textwidth}
    Fungsi tidak hanya terbatas pada satu variabel. Ada pula fungsi dengan banyak variabel. Dan ya, ada pula fungsi rekursi dengan banyak variabel. 
    
    \begin{exampleblock}{Contoh}
      Suatu fungsi didefinisikan sebagai berikut:
      \begin{itemize}
        \item $f(x,y) = 2f(x-2,y) - f(x,y-2)$
        \item $f(x,1) = f(x,0) = 1$
        \item $f(1,y) = f(0,y) = 10$
      \end{itemize}
      Tentukan nilai dari $f(5,4)$
    \end{exampleblock}
    
    
    \column{0.5\textwidth}
    \begin{block}{Jawaban}
      Kita cari satu persatu.
      \begin{itemize}
        \item $f(5,4) = 2\cdot f(3,4) - f(5,2)$
        \item $f(3,4) = 2\cdot f(1,4) - f(3,2)$
        \item $f(5,2) = 2\cdot f(3,2) - f(5,1)$
        \item $f(3,2) = 2\cdot f(1,2) - f(3,0)$
      \end{itemize}
      Sekarang kita tahu bahwa $f(1,2)=10$ dan $f(3,0)=1$. Dengan ini:
      \begin{itemize}
        \item $f(3,2) = 20 - 1 = 19$
        \item $f(5,2) = 38 - 1 = 37$
        \item $f(3,4) = 20 - 19 = 1$
        \item $f(5,4) = 2 - 37 = \boxed{-35}$
      \end{itemize}
    \end{block}
    
    
  \end{columns}
  \end{frame}
  
  \section{Fungsi Lainnya}
  \label{sec:Fungsi Lainnya}
  
  \begin{frame}{Fungsi Atap dan Lantai}
  
  \begin{columns}[c]
  \column{0.5\textwidth}
    \begin{block}{Fungsi Lantai}
      Secara umum, fungsi lantai dinyatakan dengan $\floor{x}$ yang artinya bilangan bulat terbesar yang kurang dari atau sama dengan $x$. Nilai dari $\floor{x}$ selalu berupa bilangan bulat yang memenuhi
      $$\floor{x}\leq x < \floor{x}+ 1$$
      
      Sebagai contoh:
      \begin{itemize}
        \item $\floor{3.2} = 3$
        \item $\floor{4.99} = 4$
        \item $\floor{-2.1} = -3$
      \end{itemize}
    \end{block}
    
    \column{0.5\textwidth}
    \begin{block}{Fungsi Atap}
      Secara umum, fungsi atap dinyatakan dengan $\ceil{x}$ yang artinya bilangan bulat terkecil yang lebih dari atau sama dengan $x$. Nilai dari $\ceil{x}$ selalu berupa bilangan bulat yang memenuhi
      $$\ceil{x}-1< x \leq \ceil{x}$$
      
      Sebagai contoh:
      \begin{itemize}
        \item $\ceil{3.2} = 4$
        \item $\ceil{4.99} = 5$
        \item $\ceil{-2.1} = -2$
      \end{itemize}
    \end{block}
    
    
    \end{columns}
  \end{frame}
  
  \section{Contoh Soal}
  \label{sec:Contoh Soal}
  
  \begin{frame}{Contoh Soal}
    \begin{enumerate}
      \item Suatu fungsi didefinisikan sebagai berikut
      $$f(n) =
      \begin{cases}
      2n, &\quad n \leq 2 \\
       2f(n-1) + 3f(n-2), & \quad n > 2
      \end{cases}
      $$
      Tentukan nilai dari $f(5)-f(4)$
      \item Tentukan interval $x$ yang memenuhi $\floor{x-9}+2 = 12$
    \end{enumerate}
  \end{frame}
  
  \begin{frame}{Jawaban}
    \begin{enumerate}
      \item Lakukan pengisian tabel 
      \begin{table}[!h]
        \begin{tabular}{|c|c|c|c|c|c|}
          \hline
          $n$ & 1 & 2 & 3 & 4 & 5 \\
          \hline
          $f(n)$ & 2 & 4 & 14 & 40 & 122\\ \hline
        \end{tabular}
        
        \caption{Pengisian Tabel}
      \end{table}
      Sehingga didapatlah $f(5)-f(4) = 82$
      \item Untuk menyelesaikan soal ini, kita selesaikan sebagaimana menyelesaikan pertidaksamaan.
      \begin{align*}
        \floor{x-9}+2 &= 12 \\
        \floor{x-9} &= 10 \\
        \floor{x-9} \leq &~x-9 < \floor{x-9}+1 &\quad\text{Definisi fungsi lantai}\\
        10 \leq &~x-9 < 11 &\quad\text{Substitusi nilai}\\
        19 \leq &~x < 20
      \end{align*}
      Jadi, interval yang memenuhi adalah $19 \leq x < 20$
    \end{enumerate}
    \nocite{*}
  \end{frame}
  
  \section*{Pustaka}
  \label{sec:Pustaka}
  
  \begin{frame}{Pustaka}
    \bibliography{pustaka.bib}
    \bibliographystyle{plain}
  \end{frame}
\end{document}