\documentclass[10pt,aspectratio=54, handout]{beamer}
%\usefonttheme[onlymath]{serif}
\usepackage{graphicx}
\usepackage{tikz}
\usepackage{siunitx}

\usetheme{Berkeley}
\usepackage{polyglossia}
\setdefaultlanguage[variant=indonesian]{malay}

\author[Haikal Isa]{Haikal Isa Al Mahdi}
\title{Dinamika Partikel}

\begin{document}
  \frame{\maketitle}
  \frame{
    \frametitle{Daftar Isi}
    \tableofcontents
  }
  
  \section{Partikel yang Bergerak}
  \label{sec:Pendahuluan}
  
  \begin{frame}{Dinamika Partikel}
    
    Dinamika partikel mempelajari tentang gerak benda beserta penyebabnya. Dalam dinamika partikel, bentuk benda yang mengalami gaya diabaikan. Hal ini berbeda dengan Dinamika/keseimbangan translasi dan rotasi. \\~\\
    Dinamika partikel erat kaitannya dengan Hukum Gerak Newton. Perlu diketahui bahwa gaya adalah besaran vektor. 
    \begin{block}{Intinya}
      \begin{enumerate}
        \item Bila tak ada gaya luar , $F=0,\, \Delta v = 0$
        \item Selain daripada itu, $F=ma$
        \item Jika suatu benda melakukan gaya sebesar $F$ kepada benda lainnya. Maka benda tersebut (yang terkena gaya) akan melakukan gaya sebesar $F$ namun dengan arah yang berkebalikan. $F_{\text{Aksi}}=-F_{\text{Reaksi}}$
      \end{enumerate}
    \end{block}
    
    

  
  
  
  \end{frame}
  
  \section{Diagram Benda Bebas}
  \label{sec:DIAGRAM BENDA BEBAS}
  
  \begin{frame}{Diagram Benda Bebas (DBB)}
  \begin{columns}[c]
    \column{0.4\textwidth}
    
    
  \begin{figure}[h!]
  \begin{center}
  \begin{tikzpicture}[scale=0.4]
    \draw (0,0) -- (8,0);
    \draw (3,0) rectangle (6,3);
    \draw[->](4.5,1.5) -- (4.5,-1);
    \draw[->](4,0) -- (4,5);
    
    \draw[->](0,1.5) -- (2.5,1.5);
    \draw[->,thick](4.5, 0) -- (2, 0);
    
    
    \node[below] at (4.5,-1){$mg$};
    \node[above] at (4,5){$N$};
    \node[above] at (1.25,1.5){$F$};
    \node[below] at (3,0){$f$};
    
  \end{tikzpicture}
  \end{center}
  \caption{Diagram Benda Bebas pada Benda yang Dikenai gaya $F$ pada permukaan kasar dengan gaya gesek $f$}
  \label{fig:}
  \end{figure}
  
  
    \column{0.6\textwidth}
    Adalah hal yang mendasar bagi seorang untuk mengetahui cara menggambar diagram benda bebas (DBB). Ini merupakan langkah yang penting.\\
    Macam-macam gaya sentuh:
    \begin{enumerate}
      \item Gaya Berat
      \item Gaya Normal
      \item Gaya Gesek
      \item Gaya Tegangan Tali
      \item Gaya Pegas
      \item dll.
    \end{enumerate}
    Gambar di samping adalah contoh DBB. Dalam menggambar DBB, kita menggambar seluruh gaya yang bekerja pada benda yang perlu dianalisis.
    
  \end{columns}
  
  \end{frame}
  
  \subsection{Amati, Tiru, Pelajari}
  
  \begin{frame}{Amati, Tiru, Pelajari}
  \begin{columns}[c]
    \column{0.5\textwidth}
    
    
  
    Mungkin semua sudah tahu. Rumus gaya untuk massa dan percepatan konstan adalah:
    $$F = ma$$
    dengan $m$ adalah massa dan $a$ adalah percepatan. sebagai contoh
    \begin{exampleblock}{Contoh Sederhana}
      Sebuah benda 5 \unit{kg} mula-mula diam. Abaikan gesekan. Tentukan gaya yang perlu diberikan agar kecepatan benda tersebut menjadi 7 \unit{ms^{-1}} setelah 28 detik.
    \end{exampleblock}
    \column{0.5\textwidth}
    \begin{block}{Jawaban}
      Jelaslah bahwa kita perlu mencari percepatan.
      $$a = \frac{7}{28} = 0.25~\unit{m.s^{-2}}$$
      dengan ini 
      $$F = 5\cdot 0.25 = 1.25~\unit{N}$$
      Jawabannya adalah 1.25 \unit{N} \\~\\
      Perhatikan bahwa situasi bisa saja di luar dugaan. Oleh karena itu, diingatkan bagi pembaca untuk menguasai kinematika terlebih dahulu.
    \end{block}
    
    
  \end{columns}
  
  
  \end{frame}
  
  \begin{frame}{Catatan penting}
    Mari kita lihat beberapa catatan penting:
    \begin{itemize}
      \item Gaya normal ($N$) bekerja pada benda yang bersentuhan.
      \item Gesekan terbagi menjadi gesekan statis dan kinetis
      \item Gesekan statis akan mencegah benda bergerak sampai gaya yang melawan gesekan ini sama dengan gaya gesek statis
      \item Gesekan kinetis mengurangi percepatan benda.
      \item Tali arahnya menarik
      \item Gaya gesek maksimum yang bekerja pada benda adalah $$f = \mu N$$
      dengan $\mu$ adalah koefisien gesekan dan $N$ adalah gaya normal.
    \end{itemize}
  \end{frame}
  
  \begin{frame}{Contoh DBB}
    \begin{columns}
      \column{0.5\textwidth}
      \begin{figure}
      \begin{center}
      \begin{tikzpicture}[scale=0.4]
        \draw (0,0) -- (8,0);
        \draw (1,0) -- (7,30/7) -- (7,0);
        
        \draw[rotate around={35.54:(2,5/7)}] (2,5/7) rectangle (4,3);
        \draw (3,0) arc (0:35.54:2);
        \draw[->] (2,2.4) -- (2,-1);
        \draw[->,rotate around={35.54:(2,5/7)}] (3,2) -- (0,2);
        \draw[->,rotate around={35.54:(2,5/7)}] (3,2) -- (3,4);
        
        \draw[rotate around={35.54:(2,5/7)}] (3,4) node[above,rotate=35.54] {$mg\cos \theta$};
        \draw[rotate around={35.54:(2,5/7)}] (0,2) node[left,rotate=35.54] {$mg\sin \theta$};
        
        
        \node[below] at (2,-1) {$mg$};
        
        \node[above] at (2.5,0) {$\theta$};
        
      \end{tikzpicture}
      \end{center}
      \caption{Bidang Miring}
      \end{figure}
      
      \column{0.5\textwidth}
        Pada gambar di samping, bisa dilihat gaya-gaya yang bekerja pada benda di bidang miring.
        \begin{itemize}
          \item Gaya normal sebesar $N=mg\cos \theta$
          \item percepatan benda menuruni bidang miring sebesar $g\sin \theta$
        \end{itemize}
    Contoh penyelesaian bisa dilihat di slide selanjutnya.
    
    
    
    \end{columns}
  \end{frame}
  
  
  \section{Contoh}
  \label{sec:Contoh}
  
  \begin{frame}{Contoh Soal Tipe Olimpiade}
    \begin{exampleblock}{Bidang Miring}
      \begin{figure}[!h]
      \begin{center}
          \includegraphics[scale=0.14]{diagram}
      \end{center}
      \end{figure}
      Sebuah balok bermassa $m$ ditempatkan di puncak bidang miring dengan kemiringan $\alpha$ seperti yang terlihat di gambar. Koefisien gesek kinetis dari bidang miring adalah $\mu_k$
      \begin{enumerate}
        \item Tentukan percepatan balok saat berada di bidang miring (Nyatakan dalam $m,g,\alpha,\mu_k$)
        \item Tentukan waktu yang diperlukan balok dari mulai bergerak hingga mendarat di lantai. (Nyatakan dalam $m,g,\alpha,\mu_k, h_1, h_2$)
      \end{enumerate}
    \end{exampleblock}
  \end{frame}
  
  
  \begin{frame}[shrink]{Jawaban}
    \begin{columns}[c]
      \column{0.5\textwidth}
      \begin{figure}
      \begin{center}
          \includegraphics[scale=0.17]{dbb}
      \end{center}
      \caption{DBB. Perbesar apabila kurang jelas}
      \end{figure}
      
      
      \column{0.5\textwidth}
      Mari kita mulai.
      $$ \sum{F_x} = ma$$
      $$ \sum{F_y} = 0$$
      Hal ini karena benda hanya menuruni bidang miring dan tidak melompat.
      $$W\sin \alpha - f_k = ma$$
      $$mg\sin\alpha - \mu_k N = ma$$
      $$mg\sin\alpha -\mu_k mg\cos\alpha = ma$$
      Sehingga didapat percepatannya
      $$\boxed{a = g(\sin\alpha - \mu_k\cos\alpha)}$$
    \end{columns}
    
    
  \end{frame}
  
  \begin{frame}{Latihan Untuk Pembaca}
    Untuk bagian yang kedua diserahkan kepada pembaca. Sebagai langkah awal, panjang bidang miringnya adalah 
    $$\frac{h_1}{\sin\alpha}$$
    Sisanya diserahkan kepada pembaca sebagai latihan
  \end{frame}
  
  \section{Belum Selesai}
  \label{sec:Belum Selesai}
  
  \begin{frame}{Eits! Tunggu Dulu}
  \framesubtitle{Masih ada lagi (Katrol)}
  \begin{columns}
    \column{0.5\textwidth}
    \begin{figure}
    \begin{center}
        \includegraphics[scale=0.22]{pulley}
    \end{center}
    \caption{Katrol Tetap}
    
    \end{figure}
    
    
    \column{0.5\textwidth}
    
    Katrol akan menambah hal yang baru, yaitu tegangan tali. Untuk kasus dinamika partikel, massa katrol dan tali diabaikan. Pada gambar di samping, terdapat 2 benda yang tergantung.
  \end{columns}
  
    
    
  \end{frame}
   \begin{frame}{Beberapa Hal}
    Berdasarkan gambar sebelumnya, misalkan masing-masing benda bermassa $m_A$ dan $m_B$. Katakanlah $m_B>m_A$. Artinya:
    \begin{itemize}
      \item 
    Benda benda B akan turun, sementara benda A akan naik.
    
    \item Percepatan naik benda A sama dengan percepatan turun benda B.
    \item Gaya tegangan talinya adalah $T$.
    \item dalam hal ini, $m_B g - T = m_Ba$. Sementara $T - m_A g= m_A a$ karena tarikan dari $T$ menurut benda A \textbf{arahnya sama} dengan gaya berat dari benda B.
    \item Berdasarkan kedua persamaan tersebut, percepatannya adalah $$a=\frac{m_B-m_A}{m_B+m_A}g$$
    \end{itemize}
   \end{frame}
   
   \subsection{Contoh Lagi}
   \label{sub:Contoh Lagi}
   
   \begin{frame}{Contoh lagi}
     Perhatikan gambar berikut!
     \begin{figure}[!h]
     \begin{center}
         \includegraphics[scale=0.4]{probstate}
         \caption{Katrol di bidang miring}
     \end{center}
     \end{figure}
     
     Jika $m_a=2m_b$, tentukan nilai minimal ($\theta$) agar \textbf{benda b tidak turun}. Abaikan gesekan.
   \end{frame}
   
   
   \begin{frame}{Jawaban}
   
   \framesubtitle{Mana Diagramnya?}
   \begin{columns}[c]
     \column{0.65\textwidth}
     
   
   

   
     Agar tidak turun, berlaku
     $\sum{F_a}\geq \sum{F_b}$.
     
     $$\sum{F_a}=m_a a$$
     $$m_a g\sin\theta-T = m_a a$$
     
     $$\sum{F_b}=m_b a$$
     $$m_b g-T = m_b a$$
     Artinya,
     $$m_a g\sin\theta \geq m_b g$$
     $$2m_b g\sin\theta \geq m_b g$$
     $$\sin\theta \geq \frac{1}{2}$$
     
     \column{0.35\textwidth}
     \begin{block}{}
       Bagaimana jika $m_a=km_b$ dengan k adalah konstanta serta koefisien gesek bidang miring $\mu_k$? Nyatakan jawaban dalam $\arcsin(w)$ dengan w adalah suatu konstanta.
     \end{block}
     
     
    \end{columns}
     
     Jelaslah bahwa kriteria sudutnya adalah $\boxed{\theta \geq \frac{\pi}{6}\unit{rad}}$ atau $\boxed{\theta \geq 30^\circ}$
   \end{frame}
   
   
   \section{Kerangka Non Inersial}
   \label{sec:Gaya Fiktif}
   
   \begin{frame}{Kerangka Non Inersial}
     Normalnya, kita menganalisis gaya dari acuan yang (dianggap) diam ataupun berkecepatan konstan. Kerangka acuan ini disebut Kerangka Inersial. Namun, bagaimana jika kita menganalisis gaya dari acuan yang mengalami percepatan?\\~\\
     
     Ini disebut sebagai Kerangka Non inersial. Dalam kerangka ini, Hukum Newton tidak berlaku. Agar bisa menyesuaikan, perlu ditambahkan gaya fiktif. \\~\\
     Namun, pembahasan gaya fiktif tidak akan diterangkan di sini. Ini akan diterangkan pada topik selanjutnya, Insya Allah.
   \end{frame}
   
   \frame{
     \frametitle{Tonton Juga Playlist Dinamika Benda Partikel}
     \framesubtitle{Dari Channel BengkelMaFiA}
     \nocite{*}
     \bibliography{pust_partikel.bib}
     \bibliographystyle{plain}
   }
\end{document}