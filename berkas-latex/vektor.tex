\documentclass{beamer}
\usetheme{Berkeley}
\usepackage{amsmath}


\title{Pengenalan Vektor}
\author{Haikal Isa Al Mahdi}
\date{}


\begin{document}

\begin{frame}
\titlepage
\end{frame}


\begin{frame}
\frametitle{Pengenalan Vektor}
Anggap vektor adalah alat untuk menentukan nilai dan arah suatu objek. Nilai bisa berupa posisi, kecepatan, percepatan, gaya, momentum, dll.

Vektor bisa dinyatakan sebagai berikut:
\begin{equation*}
\vec{v} = x\hat{\imath} + y\hat{\jmath} + z\hat{k}
\end{equation*}
yang panjangnya adalah:
\begin{equation*}
\|\vec{v}\| = \sqrt{v_x^2 + v_y^2 + v_z^2}
\end{equation*}
Selanjutnya kita akan mempelajari (mungkin menghafal) operasi-operasi dalam vektor.
\end{frame}

\begin{frame}{Operasi Vektor}
\begin{itemize}
   \item Penjumlahan dan Pengurangan
       \begin{equation*}
           \vec{u} \pm \vec{v} = (u_x \pm v_x)\hat{\imath} + (u_y \pm v_y)\hat{\jmath} + (u_z \pm v_z)\hat{k}
       \end{equation*}
   \item Perkalian Skalar
       \begin{equation*}
           c\vec{v} = c(v_x\hat{\imath} + v_y\hat{\jmath} + v_z\hat{k}) = cv_x\hat{\imath} + cv_y\hat{\jmath} + cv_z\hat{k}
       \end{equation*}
   \item Perkalian Titik
       \begin{equation*}
           \vec{u} \cdot \vec{v} = u_xv_x + u_yv_y + u_zv_z
       \end{equation*}
\end{itemize}
\end{frame}

\begin{frame}
\frametitle{Operasi Vektor (lanjutan)}
\begin{itemize}
   \item Perkalian Silang
       \begin{align*}
           \vec{u} \times \vec{v} &= (u_yv_z - u_zv_y)\hat{\imath} + (u_zv_x - u_xv_z)\hat{\jmath} + (u_xv_y - u_yv_x)\hat{k} \\
                                  &= \begin{vmatrix}
                                       \hat{\imath} & \hat{\jmath} & \hat{k} \\
                                       u_x & u_y & u_z \\
                                       v_x & v_y & v_z
                                     \end{vmatrix}
       \end{align*}
   \item Sudut Terkecil antara Dua Vektor
       \begin{align*}
           \vec{u} \cdot \vec{v} &= \|\vec{u}\|\|\vec{v}\| \cos\theta \\
           \vec{u} \times \vec{v} &= \|\vec{u}\|\|\vec{v}\| \sin\theta
       \end{align*}
\end{itemize}
\end{frame}

\begin{frame}
\frametitle{Contoh Soal 1}
Sebuah objek mula-mula diam. Kemudian bergerak 3 meter ke kanan dan 8 meter ke atas. Kemudian bergerak lagi 6 meter ke kiri dan 12 meter ke bawah. Dimana objek tersebut sekarang dari titik awalnya?
\end{frame}

\begin{frame}
\frametitle{Solusi Contoh Soal 1}
\begin{itemize}
   \item Bergerak 3 meter ke kanan dan 8 meter ke atas: $\vec{u} = 3\hat{\imath} + 8\hat{\jmath}$
   \item Bergerak 6 meter ke kiri dan 12 meter ke bawah: $\vec{v} = -6\hat{\imath} - 12\hat{\jmath}$
   \item Resultannya adalah $\vec{u} + \vec{v} = (3 - 6)\hat{\imath} + (8 - 12)\hat{\jmath} = -3\hat{\imath} - 4\hat{\jmath}$
\end{itemize}
Sehingga objek tersebut berada pada 3 meter ke kiri dan 4 meter ke bawah dari titik awal.
\end{frame}

\begin{frame}
\frametitle{Contoh Soal 2}
Dua Mobil mula-mula berada titik yang sama $(0, 0)$. Kedua mobil bergerak lurus namun berbeda arah. Mobil A berhenti di titik $(6, 8)$. Mobil B berhenti titik $(7, 24)$. Tentukan sudut terkecil yang dibentuk dari lintasan kedua mobil tersebut.
\end{frame}

\begin{frame}
\frametitle{Solusi Contoh Soal 2}
\begin{itemize}
   \item Vektor lintasan mobil A: $\vec{v} = 6\hat{\imath} + 8\hat{\jmath}$
   \item Vektor lintasan mobil B: $\vec{w} = 7\hat{\imath} + 24\hat{\jmath}$
   \item Panjang vektor $\vec{v}$ dan $\vec{w}$:
       \begin{align*}
           \|\vec{v}\| &= \sqrt{6^2 + 8^2} = 10 \\
           \|\vec{w}\| &= \sqrt{7^2 + 24^2} = 25
       \end{align*}
   \item Menggunakan perkalian titik:
       \begin{align*}
           \vec{v} \cdot \vec{w} &= \|\vec{v}\|\|\vec{w}\| \cos\theta \\
                                  &= 42 + 192 = 250 \cos\theta \\
                                  &\implies \cos\theta = 0.936 \\
                                  &\implies \theta \approx 20.61^\circ
       \end{align*}
\end{itemize}
Jadi, sudut terkecil yang dibentuk dari lintasan kedua mobil tersebut adalah $20.61^\circ$.
\end{frame}

\begin{frame}
\frametitle{Soal}
\begin{enumerate}
   \item Tentukan nilai dari $r\vec{v}$ untuk $r = 1.4$ dan $\vec{v} = 7\hat{\imath} + \pi\hat{\jmath}$.
   \item Diketahui $\vec{v} = 3\hat{\imath} + 4\hat{\jmath} + \hat{k}$ dan $\vec{w} = 4\hat{\imath} + 7\hat{\jmath} - 2\hat{k}$. Tentukan nilai dari:
       \begin{enumerate}
           \item $\vec{v} \cdot \vec{w}$
           \item $\vec{v} \times \vec{w}$
       \end{enumerate}
   \item Tentukan luas dari segitiga yang dibentuk oleh ketiga titik berikut: $(0, 0)$, $(4, 8)$, dan $(10, 7)$.
   
   \end{enumerate}
   \end{frame}
   
   
 

\begin{frame}[allowframebreaks]
\frametitle{Kunci Jawaban}
\begin{enumerate}
   \item $r\vec{v} = 1.4(7\hat{\imath} + \pi\hat{\jmath}) = 9.8\hat{\imath} + 1.4\pi\hat{\jmath}$
   \item \begin{enumerate}
           \item $\vec{v} \cdot \vec{w} = 3(4) + 4(7) + 1(-2) = 38$
           \item $\vec{v} \times \vec{w} = \begin{vmatrix}
                                       \hat{\imath} & \hat{\jmath} & \hat{k} \\
                                       3 & 4 & 1 \\
                                       4 & 7 & -2
                                     \end{vmatrix} = -13\hat{\imath} + 10\hat{\jmath} + 5\hat{k}$
            \item $\sqrt{26}$
       \end{enumerate}
   \item Misalkan titik $(0, 0)$, $(4, 8)$, dan $(10, 7)$ berturut-turut $A$, $B$, dan $C$. Vektor $\vec{AB} = 4\hat{\imath} + 8\hat{\jmath}$ dan $\vec{AC} = 10\hat{\imath} + 7\hat{\jmath}$. Luas segitiga:
       \begin{align*}
           \text{Luas} &= \frac{1}{2}|\vec{AB} \times \vec{AC}| \\
                      &= \frac{1}{2}\left|\begin{vmatrix}
                                       \hat{\imath} & \hat{\jmath} & \hat{k} \\
                                       4 & 8 & 0 \\
                                       10 & 7 & 0
                                     \end{vmatrix}\right| \\
                      &= \frac{1}{2}|(28 - 80)\hat{k}| \\
                      &= \frac{1}{2}|28 - 80| = 26
       \end{align*}
\end{enumerate}
\end{frame}

\end{document}